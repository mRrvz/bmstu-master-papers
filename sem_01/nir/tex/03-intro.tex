\section*{ВВЕДЕНИЕ}
\addcontentsline{toc}{section}{ВВЕДЕНИЕ}

Динамическая память -- ценный вычислительный ресурс, необходимый для управления всей системой. Производительность системы зависит от эффективности управления динамической памятью. Поэтому, все современные многозадачные операционные системы пытаются оптимизировать использование оперативной памяти выделяя ее только в случае необходимости и освобождая ее при первой же возможности. Для достижения данной цели, ядро операционной системы использует специальные алгоритмы и структуры данных. Цель данной работы -- классификация методов распределения памяти в ядре Linux.

Для достижения поставленной цели необходимо решить следующие задачи:

\begin{itemize}
	\item провести обзор существующих методов распределения памяти в ядре Linux;
	\item описать плюсы и недостатки каждого из методов;
	\item сформулировать критерии классификации;
	\item классифицировать существующие методы распределения памяти;
\end{itemize}

\pagebreak
