\section*{ЗАКЛЮЧЕНИЕ}
\addcontentsline{toc}{section}{ЗАКЛЮЧЕНИЕ}

В ходе выполнения научно исследовательский работы была достигнута ее цель --  классифицированы методы распределения памяти в ядре Linux.

Для достижения данной цели были решены следующие задачи:

\begin{itemize}
	\item проведен обзор существующих методов распределения памяти в ядре Linux;
	\item описать плюсы и недостатки каждого из методов;
	\item сформулированы критерии классификации;
	\item классифицированы существующие методы распределения памяти;
\end{itemize}

Сделан вывод, что каждый метод необходим для решения конкретной задачи. Таким образом, универсального метода, который был бы применим во всех ситуациях, не существует. Опираясь на классификацию методов, данную в этой научно-исследовательской работе, можно делать выводы о применимости алгоритма для той или иной задачи.

\pagebreak