\section*{ВВЕДЕНИЕ}
\addcontentsline{toc}{section}{ВВЕДЕНИЕ}

Необходимость повышения безопасности исполнения приложений, работающих в системах безопасности и обрабатывающих защищаемую информацию, привела к разработке программно-аппаратных решений, создающих доверенные среды исполнения (англ. TEE -- Trusted Execution Environment \cite{tee}) на базе аппаратных средств, доверенных загрузок или аппаратно-программных модулей доверенной загрузки. Intel \cite{intel} и ARM \cite{arm} являются лидерами в этой области. Целью данной работы является анализ и сравнение существующих реализаций доверенных сред исполнения (ДСИ).

Для достижения поставленной цели необходимо решить следующие задачи:

\begin{itemize}
	\item провести обзор существующих реализаций ДСИ;
	\item описать достоинства и недостатки каждой из реализаций;
	\item сформулировать критерии сравнения;
	\item сравнить существующие реализации.
\end{itemize}

\pagebreak
