\section*{ВВЕДЕНИЕ}
\addcontentsline{toc}{section}{ВВЕДЕНИЕ}

Необходимость повышения безопасности исполнения приложений, работающих в системах безопасности и обрабатывающих защищаемую информацию, привела к разработке программно-аппаратных решений, создающих доверенные среды исполнения (англ. TEE -- Trusted Execution Environment \cite{tee}) на базе аппаратных средств, доверенных загрузок или аппаратно-программных модулей доверенной загрузки. Intel \cite{intel} и ARM \cite{arm} являются лидерами в этой области. Целью данной работы является исследование эффективности и применимости метода программной реализации доверенной среды исполнения с помощью виртуализации процессоров архитектуры ARM.

Для достижения поставленной цели необходимо решить следующие задачи:

\begin{itemize}
	\item провести исследование эффективности и применимости ПО;
	\item выполнить сравнение метода программной реализации с аппаратным методом.
\end{itemize}

\pagebreak
