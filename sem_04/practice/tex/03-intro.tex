\section*{ВВЕДЕНИЕ}
\addcontentsline{toc}{section}{ВВЕДЕНИЕ}

\textbf{Цель проведения практики:}

Осуществление профессионально-практической подготовки студентов к будущей профессиональной деятельности, закрепление и углубление теоретических знаний в области проектирования информационных систем, овладение навыками воспринимать профессиональные знания и применять их для решения нестандартных задач, в том числе в новой или незнакомой среде при выполнении выпускной квалификационной работы магистра.

\textbf{Задачи проведения практики:}

Выполнить оформление полученных результатов научных исследований по теме ВКР <<Метод программной реализации доверенной среды исполнения с помощью виртуализации процессоров архитектуры ARM>>  в виде разделов РПЗ  выпускной квалификационной работы магистра:

\begin{itemize}
	\item [---] Аннотация ВКР.
	\item [---] Оглавление ВКР.
	\item [---] Конструкторский раздел ВКР.
	\item [---] Исследовательский раздел ВКР.
	\item [---] Список использованных источников в ВКР.
\end{itemize}

""

\begin{itemize}
\item Вид практики – производственная.
\item Способы проведения практики – стационарная (МГТУ им. Н.Э. Баумана)
\item Форма проведения – распределенная – проходит в течение семестра.
\item Тип практики – преддипломная практика.
\end{itemize}

\pagebreak
